% PRL look and style (easy on the eyes)
\documentclass[aps,pre,twocolumn,nofootinbib,superscriptaddress,linenumbers]{revtex4-1}
% Two-column style (for submission/review/editing)
%\documentclass[aps,prl,preprint,nofootinbib,superscriptaddress,linenumbers]{revtex4-1}

\pdfoutput=1
\usepackage[pdftex]{graphicx}

%\usepackage{palatino}

%\usepackage{palatino}
% Change to a sans serif font.
\usepackage{sourcesanspro}
\renewcommand*\familydefault{\sfdefault} %% Only if the base font of the document is to be sans serif
\usepackage[T1]{fontenc}
%\usepackage[font=sf,justification=justified]{caption}
\usepackage[font=sf]{floatrow}

% Rework captions to use sans serif font.
\makeatletter
\renewcommand\@make@capt@title[2]{%
 \@ifx@empty\float@link{\@firstofone}{\expandafter\href\expandafter{\float@link}}%
  {\sf\textbf{#1}}\sf\@caption@fignum@sep#2\quad
}%
\makeatother

\usepackage{listings} % For code examples
\usepackage[usenames,dvipsnames,svgnames,table]{xcolor}

\usepackage{amsmath}
\usepackage{amssymb}
%\usepackage[mathbf,mathcal]{euler}
%\usepackage{citesort}
\usepackage[caption=false]{subfig}
\usepackage{dcolumn}
\usepackage{boxedminipage}
\usepackage{verbatim}
\usepackage[colorlinks=true,citecolor=blue,linkcolor=blue]{hyperref}
\usepackage[group-separator={,}]{siunitx}

% Justification
\captionsetup{singlelinecheck=off}

% Pretty-printing of shell commands
\newcommand{\shellcmd}[1]{\\\ \texttt{\scriptsize\# #1}\\}

% The figures are in a figures/ subdirectory.
\graphicspath{{../figures/}}


% \newcommand{\pyitc}{\url{http://www.simtk.org/home/bayesian-itc}} % URL of pyITC project homepage

%% DOCUMENT %%%%%%%%%%%%%%%%%%%%%%%%%%%%%%%%%%%%%%%%%%%%%%%%%%%%%%%%%%%%%%%%%%%%
\begin{document}

%% TITLE %%%%%%%%%%%%%%%%%%%%%%%%%%%%%%%%%%%%%%%%%%%%%%%%%%%%%%%%%%%%%%%%%%%%
\title{Supporting Information for "Ensembler: Enabling high-throughput molecular simulations at the superfamily scale"}

\author{Daniel L. Parton}
  \affiliation{Computational Biology Program, Sloan Kettering Institute, Memorial Sloan Kettering Cancer Center, New York, NY 10065}
  %\email{daniel.parton@choderalab.org}
\author{Patrick B. Grinaway}
  \affiliation{Computational Biology Program, Sloan Kettering Institute, Memorial Sloan Kettering Cancer Center, New York, NY 10065}
  %\email{patrick.grinaway@choderalab.org}
\author{John D. Chodera}
 \thanks{Corresponding author}
 \email{john.chodera@choderalab.org}
  \affiliation{Computational Biology Program, Sloan Kettering Institute, Memorial Sloan Kettering Cancer Center, New York, NY 10065}

\date{\today}

\maketitle

%%%%%%%%%%%%%%%%%%%%%%%%%%%%%%%%%%%%%%%%%%%%%%%%%%%%%%%%%%%%%%%%%%%%%%%%%%%%%%%%%%%%%%%%%%%%%%%%%%%%%%
% MAIN
%%%%%%%%%%%%%%%%%%%%%%%%%%%%%%%%%%%%%%%%%%%%%%%%%%%%%%%%%%%%%%%%%%%%%%%%%%%%%%%%%%%%%%%%%%%%%%%%%%%%%%

\begin{figure*}[tb]
    \includegraphics[width=0.7\textwidth]{loopmodel_analysis/nmissing_resis_distributions}

  \caption{{\bf Distributions of the number of missing residues for templates for which remodeling (with the {\tt loopmodel} command) was either successful or unsuccessful.}
  The plotted distributions are smoothed using kernel density estimation, and the raw data points are shown as a rug plot.
  }
  \label{figpipeline}
\end{figure*}

%%%%%%%%%%%%%%%%%%%%%%%%%%%%%%%%%%%%%%%%%%%%%%%%%%%%%%%%%%%%%%%%%%%%%%%%%%%%%%%%%%%%%%%%%%%%%%%%%%%%%%
% BIBLIOGRAPHY
%%%%%%%%%%%%%%%%%%%%%%%%%%%%%%%%%%%%%%%%%%%%%%%%%%%%%%%%%%%%%%%%%%%%%%%%%%%%%%%%%%%%%%%%%%%%%%%%%%%%%%

%\bibliographystyle{prsty} 
\bibliography{ms.bib}

\end{document}
