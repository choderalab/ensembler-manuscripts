% PRL look and style (easy on the eyes)
\documentclass[aps,pre,twocolumn,nofootinbib,superscriptaddress,linenumbers]{revtex4-1}
% Two-column style (for submission/review/editing)
%\documentclass[aps,prl,preprint,nofootinbib,superscriptaddress,linenumbers]{revtex4-1}

\usepackage{palatino}
\usepackage{amsmath}
\usepackage{amssymb}
\pdfoutput=1
\usepackage[pdftex]{graphicx}
%\usepackage[mathbf,mathcal]{euler}
%\usepackage{citesort}
\usepackage{dcolumn}
\usepackage{boxedminipage}
\usepackage{verbatim}
\usepackage[colorlinks=true,citecolor=blue,linkcolor=blue]{hyperref}


% The figures are in a figures/ subdirectory.
\graphicspath{{figures/}}


% \newcommand{\pyitc}{\url{http://www.simtk.org/home/bayesian-itc}} % URL of pyITC project homepage

%% DOCUMENT %%%%%%%%%%%%%%%%%%%%%%%%%%%%%%%%%%%%%%%%%%%%%%%%%%%%%%%%%%%%%%%%%%%%
\begin{document}

%% TITLE %%%%%%%%%%%%%%%%%%%%%%%%%%%%%%%%%%%%%%%%%%%%%%%%%%%%%%%%%%%%%%%%%%%%
\title{Ensembler: Enabling high-throughput molecular simulations at the superfamily scale}

\author{Daniel L. Parton}
  \affiliation{Computational Biology Center, Memorial Sloan Kettering Cancer Center, New York, NY 10065}
  %\email{daniel.parton@choderalab.org}
\author{Patrick B. Grinaway}
  \affiliation{Computational Biology Center, Memorial Sloan Kettering Cancer Center, New York, NY 10065}
  %\email{patrick.grinaway@choderalab.org}
\author{John D. Chodera}
 \thanks{Corresponding author}
 \email{john.chodera@choderalab.org}
  \affiliation{Computational Biology Center, Memorial Sloan Kettering Cancer Center, New York, NY 10065}

\date{\today}

%%%%%%%%%%%%%%%%%%%%%%%%%%%%%%%%%%%%%%%%%%%%%%%%%%%%%%%%%%%%%%%%%%%%%%%%%%%%%%%%%%%%%%%%%%%%%%%%%%%%%%
% ABSTRACT/pacs
%%%%%%%%%%%%%%%%%%%%%%%%%%%%%%%%%%%%%%%%%%%%%%%%%%%%%%%%%%%%%%%%%%%%%%%%%%%%%%%%%%%%%%%%%%%%%%%%%%%%%%
\begin{abstract}

The rapidly expanding body of available genomic and protein structural data
provides a rich resource for the field of biomolecular simulation. However,
simulations on an omics scale are not yet widely performed, partly because
software has had trouble keeping pace. For example, it should be possible to
study proteins across entire (super)families, and to do so in a way which
exploits the entire variety of available structural biology data. Here, we
present a new tool for enabling high-throughput simulation in the genomics era.
Ensembler takes any set of sequences - from a single sequence to an entire
superfamily of interest - and shepherds them through various stages of:
comparative modeling to all relevant PDB structures, reconstruction of missing
loops, addition of missing atoms, culling by close structural similarity,
assignment of protonation states, solvation, and refinement with molecular
simulation. The output is an ensemble of structures ready for subsequent
parallel or distributed molecular simulations. This automates much of the
time-consuming process of preparing protein models suitable for simulation,
while also allowing this process to be scaled to the superfamily scale.  A
particular advantage of this approach can be found in the construction of
kinetic models of conformational dynamics - such as Markov state models - for
which a diverse array of starting configurations is expected to aid sampling.
We demonstrate the power of this approach by constructing initial models for
all catalytic domains in the human tyrosine kinase family, using all kinase
catalytic domain structures from any organism as structural templates.
Ensembler should run on all major operating systems, and has been tested on
Linux and OS X. The program is free of charge, and is made available under the
terms of the GNU General Public License (GPL) v2. The latest release can be
installed via the conda package manager, and the latest source can be
downloaded from \url{https://github.com/choderalab/ensembler}.

% TODO expand to all human protein kinases?

\end{abstract}

\maketitle

%%%%%%%%%%%%%%%%%%%%%%%%%%%%%%%%%%%%%%%%%%%%%%%%%%%%%%%%%%%%%%%%%%%%%%%%%%%%%%%%%%%%%%%%%%%%%%%%%%%%%%
% INTRODUCTION
%%%%%%%%%%%%%%%%%%%%%%%%%%%%%%%%%%%%%%%%%%%%%%%%%%%%%%%%%%%%%%%%%%%%%%%%%%%%%%%%%%%%%%%%%%%%%%%%%%%%%%
\section{Introduction}
\label{section:introduction}

%%%%%%%%%%%%%%%%%%%%%%%%%%%%%%%%%%%%%%%%%%%%%%%%%%%%%%%%%%%%%%%%%%%%%%%%%%%%%%%%%%%%%%%%%%%%%%%%%%%%%
% DESIGN AND IMPLEMENTATION
%%%%%%%%%%%%%%%%%%%%%%%%%%%%%%%%%%%%%%%%%%%%%%%%%%%%%%%%%%%%%%%%%%%%%%%%%%%%%%%%%%%%%%%%%%%%%%%%%%%%%
\section{Design and Implementation}
\label{section:design}

%%%%%%%%%%%%%%%%%%%%%%%%%%%%%%%%%%%%%%%%%%%%%%%%%%%%%%%%%%%%%%%%%%%%%%%%%%%%%%%%%%%%%%%%%%%%%%%%%%%%%
% RESULTS
%%%%%%%%%%%%%%%%%%%%%%%%%%%%%%%%%%%%%%%%%%%%%%%%%%%%%%%%%%%%%%%%%%%%%%%%%%%%%%%%%%%%%%%%%%%%%%%%%%%%%
\section{Results}
\label{section:results}

%%%%%%%%%%%%%%%%%%%%%%%%%%%%%%%%%%%%%%%%%%%%%%%%%%%%%%%%%%%%%%%%%%%%%%%%%%%%%%%%%%%%%%%%%%%%%%%%%%%%%
% DISCUSSION
%%%%%%%%%%%%%%%%%%%%%%%%%%%%%%%%%%%%%%%%%%%%%%%%%%%%%%%%%%%%%%%%%%%%%%%%%%%%%%%%%%%%%%%%%%%%%%%%%%%%%
\section{Availability and Future Directions}
\label{section:availability}

%%%%%%%%%%%%%%%%%%%%%%%%%%%%%%%%%%%%%%%%%%%%%%%%%%%%%%%%%%%%%%%%%%%%%%%%%%%%%%%%%%%%%%%%%%%%%%%%%%%%%
% ACKNOWLEDGMENTS
%%%%%%%%%%%%%%%%%%%%%%%%%%%%%%%%%%%%%%%%%%%%%%%%%%%%%%%%%%%%%%%%%%%%%%%%%%%%%%%%%%%%%%%%%%%%%%%%%%%%%
\section{Acknowledgments}
\label{section:acknowledgments}

%%%%%%%%%%%%%%%%%%%%%%%%%%%%%%%%%%%%%%%%%%%%%%%%%%%%%%%%%%%%%%%%%%%%%%%%%%%%%%%%%%%%%%%%%%%%%%%%%%%%%%
% BIBLIOGRAPHY
%%%%%%%%%%%%%%%%%%%%%%%%%%%%%%%%%%%%%%%%%%%%%%%%%%%%%%%%%%%%%%%%%%%%%%%%%%%%%%%%%%%%%%%%%%%%%%%%%%%%%%

%\bibliographystyle{prsty} 
\bibliography{ms.bib}

\end{document}
